%% This file is generated by Jinja2
% LaTeX resume using res.cls
\documentclass[line,11pt,letter]{includes/cls/myRes}

%% input all macros files
\input{includes/macros/packages.tex}
\input{includes/macros/customized_commands.tex}
\usepackage{geometry}
\geometry{
  body={7.3in, 10.4in},
  left=0.6in,
  top=0.2in
}


\begin{document}

%% personal information
\MyName{Long Gong}{Personal Website: \url{https://lgong30.github.io}}

\address{266 Ferst Dr, Atlanta, GA 30332, United States}
\address{Cell: +1(404)697-0608. \hspace{5pt} Email: gonglong@gatech.edu}
\begin{resume}

%% objective
\section{Objective}
\vspace{-4pt}
{\setlength{\parskip}{0pt}
 {Internship position in the field of networking (with special interest in software defined networking and data center networking), scheduling in switches, and software engineering. \hfill{Available: Summer and Fall 2018}\break}
}%%
%% education
\negspace
%% template for education section
\section{Education}
\vspace{-4pt}
{\setlength{\parskip}{0pt}
\textbf{Georgia Institute of Technology}, Atlanta, GA, USA\\
{\hspace*{1em} Ph.D. Candidate in Computer Science (GPA: 3.91/4.0) \hspace{52.5pt} \hfill 2015.8 - 2020.5 (\textbf{Expected})\break}
}
\sspace

{\setlength{\parskip}{0pt}
\textbf{University of Science and Technology of China}, Hefei, Anhui, China\\
{\hspace*{1em} M.Eng. in Communication and Information Systems (GPA: 3.81/4.3) \hspace{52.5pt} \hfill 2012.9 - 2015.6\break}
{\hspace*{1em} B.Eng. in Electronic Information Engineering (GPA: 3.75/4.3) \hspace{52.5pt} \hfill 2008.9 - 2012.6\break}
}
\negspace
%%
%% experience
\section{Intern Experience}
\vspace{-4pt}
{\setlength{\parskip}{0pt}
{\bf Alibaba Group (U.S.) Inc}, Bellevue, WA, USA \hfill 2018.5 - 2018.8\break
{\hspace*{1em} Intern \hfill Mentor: Gang Cheng\break}
%% @12142018 change \break at the end of the next line to \hfill\break
{\hspace*{1em} Built a highly scalable multi-tenant BGP as an important part of a high-performance and high-availability SDN based hybrid cloud network solution.\hfill\break}
}\sspace

{\setlength{\parskip}{0pt}
{\bf AT\&T Labs Research}, Bedminster, NJ, USA \hfill 2016.5 - 2016.7\break
{\hspace*{1em} Research Intern \hfill Mentor: He Yan and Zihui Ge\break}
%% @12142018 change \break at the end of the next line to \hfill\break
{\hspace*{1em} Developed tools to automate the dynamics analysis in services supported by virtualized environment.\hfill\break}
}\negspace
%%
%% project
\section{Projects}
\vspace{-4pt}
{\setlength{\parskip}{0pt}
{\bf Crossbar Scheduling} \hfill 2016.2 - Present\break
{\hspace*{1em} $\bullet$ Designed the {\bf first} parallel iterative crossbar scheduling which has constant per-port time complexity and can provably achieve the same performance (both throughput and delay) as the family of maximal matching based schedulers that have at least logarithmic per-port time complexity.\hfill\break}
{\hspace*{1em} $\bullet$ Designed a crossbar scheduling algorithm, which can exactly emulate the linear-time centralized version ({\it i.e.,} SERENA) in logarithmic rounds. (IEEE/ACM Transactions on Networking under submission)\hfill\break}
{\hspace*{1em} $\bullet$ Designed a simple yet effective ``add-on'' crossbar scheduling algorithm for input-queued switches, which can boost the performance (switch throughput or delay or both) of existing crossbar scheduling algorithms ({\it e.g.}, iSLIP and SERENA) with almost ``no'' computational overhead. (SIGMETRICS 2017)\hfill\break}
{\hspace*{1em} $\bullet$ Built an efficient and flexible input-queued switch simulator in C++.\hfill\break}
\sspace

}
{\setlength{\parskip}{0pt}
{\bf Time Capsule for Online Social Activities} \hfill 2015.9 - Present\break
{\hspace*{1em} $\bullet$ Designed a hybrid streaming-sampling algorithm for high accurate measurements of Online Social Networking (OSN) cascade statistics, using limited memory, which decreased the errors (measured in $\ell_2$) by more than one order of magnitude. (ICCCN 2017)\hfill\break}
\sspace

}
{\setlength{\parskip}{0pt}
{\bf Network Virtualization} \hfill 2012.2 - 2015.6\break
{\hspace*{1em} $\bullet$ Proposed the {\bf first} integer linear programming formulations for the virtual optical network embedding problems in the contexts of flexible-grid elastic optical networks, and designed efficient algorithms which achieved much better performance. (Journal of Lightwave Technology)\hfill\break}
{\hspace*{1em} $\bullet$ Built the {\bf first} OpenFlow-based network virtualization platform in which the underlying infrastructure is the flexible-grid elastic optical networks. (Master Thesis)\hfill\break}
}
\negspace
%% skills
\section{Professional Skills}
\vspace{-4pt}
\setlength{\parskip}{0pt}
{\sl Programming Languages:} \textsc{C++} (proficient), \textsc{Python} (fluent), \textsc{Java} (prior experience)\\
\end{resume}

\end{document}