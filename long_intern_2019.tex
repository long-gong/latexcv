%% This file is generated by Jinja2
% LaTeX resume using res.cls
\documentclass[line,11pt,letter]{/home/saber/GitHub/personal-resume/includes/cls/myRes}

%% input all macros files
%-----------------------------------------------------------
%Custom commands
\newcommand{\resitem}[1]{\item #1 \vspace{-4pt}}


\newcommand{\resheading}[1]{\vspace{8pt}
  \parbox{\textwidth}{\setlength{\FrameSep}{\outerbordwidth}
    \begin{shaded}
\setlength{\fboxsep}{0pt}\framebox[\textwidth][l]{\setlength{\fboxsep}{4pt}\fcolorbox{shadecolorB}{shadecolorB}{\textbf{\sffamily{\mbox{~}\makebox[6.762in][l]{\large \uppercase{#1}} \vphantom{p\^{E}}}}}}
    \end{shaded}
  }\vspace{-5pt}
}

\newcommand{\ressubheading}[4]{
\begin{tabular*}{7.0in}{l@{\cftdotfill{\cftsecdotsep}\extracolsep{\fill}}r}
    {\textbf{#1}} & {\textbf{#2}} \\
    {\textbf{#3}} & {\textit{#4}}
\end{tabular*}\vspace{-6pt}}
%-----------------------------------------------------------

% \newcommand{\MyName}[2]{
%         \Huge \usefont{OT1}{phv}{b}{n} #1 \hfill \large \usefont{OT1}{phv}{m}{n} \textit{#2}% Name
%         \par \normalsize \normalfont}

\newcommand{\MyName}[2]{
        \Huge \usefont{OT1}{phv}{b}{n} #1 \hfill \small \usefont{OT1}{phv}{m}{n} #2% Name
        \par \normalsize \normalfont}

\newcommand{\MySlogan}[1]{
        \large \usefont{OT1}{phv}{m}{n} \textit{#1}\hfill % Slogan (optional)
        \par \normalsize \normalfont}

\newsectionwidth{0cm}

\newcommand{\sspace}{\vspace*{-8pt}}
\newcommand{\negspace}{\vspace*{-28pt}}
%letter, 8.5in*11.0in, 11pt
\usepackage{geometry}
\geometry{
  body={6.9in, 10.0in},
  left=0.8in,
  top=0.4in
}
\usepackage{amsmath,amsfonts,amsthm,amssymb}
\usepackage{graphicx}
\usepackage[svgnames]{xcolor}

% \usepackage{hyperref}
\usepackage{url}
\usepackage{enumitem}
\usepackage{calc}
\usepackage{etoolbox}
% added @2016-11-12
\usepackage{natbib}
\usepackage{bibentry}
\usepackage[colorlinks=true,urlcolor=blue]{hyperref}



\usepackage{sectsty}    % Custom sectioning
\sectionfont{%          % Change font of \section command
    \bf %           % bch-b-n: CharterBT-Bold font
    \sectionrule{0pt}{0pt}{-5pt}{1pt}
    }

% added @2017-09-01
\usepackage{fancyhdr}
\usepackage{lastpage}
 
\pagestyle{fancy}
\fancyhf{}
\renewcommand{\headrulewidth}{0pt}
% \rfoot{Page \thepage \hspace{1pt} of \pageref*{LastPage}}

\begin{document}

%% personal information
\MyName{Long Gong}{Personal Website: \url{https://lgong30.github.io}}

\address{266 Ferst Dr, Atlanta, GA 30332, United States}
\address{Cell: +1(404)697-0608. \hspace{5pt} Email: gonglong@gatech.edu}
\begin{resume}

%% include bibfile
\nobibliography{./data/long_gong_pub.bib}
\bibliographystyle{plain}
\vspace*{-10pt}
\vspace*{-15pt}
%% objective
\section{Objective}
\vspace{-4pt}
{\setlength{\parskip}{0pt}
 {Internship position in the field of networking (with special interest in software defined networking and data center networking), scheduling in switches, and software engineering. \hfill{Available: Summer 2019}\break}
}%%
%% education
\negspace
%% template for education section
\section{Education}
\vspace{-4pt}
{\setlength{\parskip}{0pt}
\textbf{Georgia Institute of Technology}, Atlanta, GA, USA\\
{\hspace*{1em} Ph.D. Candidate in Computer Science (GPA: 3.92/4.0) \hspace{52.5pt} \hfill 2015.8 - 2020.5 (\textbf{Expected})\break}
}
\sspace

{\setlength{\parskip}{0pt}
\textbf{University of Science and Technology of China}, Hefei, Anhui, China\\
{\hspace*{1em} M.Eng. in Communication and Information Systems (GPA: 3.81/4.3) \hspace{52.5pt} \hfill 2012.9 - 2015.6\break}
{\hspace*{1em} B.Eng. in Electronic Information Engineering (GPA: 3.75/4.3) \hspace{52.5pt} \hfill 2008.9 - 2012.6\break}
}
\negspace
%%
%% experience
\section{Intern Experience}
\vspace{-4pt}
{\setlength{\parskip}{0pt}
{\bf Alibaba Group (U.S.) Inc}, Bellevue, WA, USA \hfill 2018.5 - 2018.8\break
{\hspace*{1em} Intern  \hfill Mentor: Gang Cheng\break}
{\hspace*{1em} Built a highly scalable multi-tenant BGP tool as an important component of a high-performance and high-\hfill\break availability SDN based hybrid cloud network solution.\hfill\break}
}

\sspace
% \sspace
{\setlength{\parskip}{0pt}
{\bf AT\&T Labs Research}, Bedminster, NJ, USA \hfill 2016.5 - 2016.7\break
{\hspace*{1em} Research Intern \hfill Mentor: He Yan and Zihui Ge\break}
{\hspace*{1em} Developed tools to automate the dynamics analysis in services provided by virtualized network functions.\hfill\break}
}\negspace
%%
%% project
\section{Projects}
\vspace{-4pt}
{\setlength{\parskip}{0pt}
{\bf Crossbar Scheduling} \hfill 2016.2 - Present\break
{\hspace*{1em} $\bullet$ Designed the {\bf first} parallel iterative crossbar scheduling algorithm which has constant per-port time complexity and can provably achieve the same performance (both throughput and delay) guarantees as the family of maximal matching based schedulers that have at least logarithmic per-port time complexity.\hfill\break}
{\hspace*{1em} $\bullet$ Designed a crossbar scheduling algorithm that can exactly emulate the linear-time centralized counterpart ({\it i.e.,} SERENA) in logarithmic rounds. (IEEE/ACM Transactions on Networking under submission)\hfill\break}
{\hspace*{1em} $\bullet$ Designed a simple yet effective ``add-on'' crossbar scheduling algorithm for input-queued switches, which can boost the performance (throughput or delay or both) of existing crossbar scheduling algorithms ({\it e.g.}, iSLIP and SERENA) with almost ``no'' computational overhead. (SIGMETRICS 2017)\hfill\break}
{\hspace*{1em} $\bullet$ Built an efficient and flexible input-queued switch simulator in C++.\hfill\break}
\sspace

}
{\setlength{\parskip}{0pt}
{\bf Time Capsule for Online Social Activities} \hfill 2015.9 - Present\break
{\hspace*{1em} $\bullet$ Designed a hybrid streaming-sampling algorithm for high accurate measurements of Online Social Networking (OSN) cascade statistics, using limited memory, which decreased the errors (measured in $\ell_2$) by more than one order of magnitude. (ICCCN 2017)\hfill\break}
\sspace

}
{\setlength{\parskip}{0pt}
{\bf Network Virtualization} \hfill 2012.2 - 2015.6\break
% {\hspace*{1em} $\bullet$ Designed a novel virtual network embedding algorithm, which increased the network utilization and decreased the time complexity. (INFOCOM 2014)\hfill\break}
% {\hspace*{1em} $\bullet$ Proposed the {\bf first} integer linear programming formulations for the virtual optical network embedding problems in the contexts of flexible-grid elastic optical networks, and designed efficient algorithms which achieved much better performance. (Journal of Lightwave Technology)\hfill\break}
{\hspace*{1em} $\bullet$ Proved the {\bf first} inapproximability result of the location-constrained virtual network embedding (LC-VNE) problems, and designed efficient algorithms for solving LC-VNE which have much better performance (in terms of both resource consumption and fairness) than existing ones. (IEEE/ACM Transactions on Networking)\hfill\break}
{\hspace*{1em} $\bullet$ Built the {\bf first} OpenFlow-based network virtualization platform in which the underlying infrastructure is the flexible-grid elastic optical networks. (Master Thesis)\hfill\break}
}
\negspace
%%
%% talks
\section{Professional Skills}
\vspace{-4pt}
\setlength{\parskip}{0pt}
{\sl Programming Languages:} \textsc{C++} (proficient), \textsc{Python} (fluent), \textsc{Java} (prior experience)\\
\vspace*{4pt}
\negspace
%%
%% service
% \section{Professional Service}
% \vspace{-4pt}
% \setlength{\parskip}{0pt}
% {\sl Reviewer:} \textsc{IEEE INFOCOM 2016}, \textsc{IEEE Communication Letters}, \textsc{IEEE Transactions on Parallel and Distributed Systems}, \textsc{IEEE/ACM Transactions on Networking}, \textsc{IEEE/OSA Journal of Optical Communications and Networking}\\
% 
\end{resume}

\end{document}





